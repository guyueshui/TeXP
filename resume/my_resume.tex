% !TEX program = xelatex
\documentclass{resume}
%\usepackage{zh_CN-Adobefonts_external} % Simplified Chinese Support using external fonts (./fonts/zh_CN-Adobe/)
%\usepackage{zh_CN-Adobefonts_internal} % Simplified Chinese Support using system fonts
\usepackage[typography]{../utils/zhfont}

% set pdf metadata
\usepackage{hyperref}
\hypersetup{
  pdftitle={蔡和章的简历},
  pdfauthor={蔡和章},
  pdfsubject={简历},
  pdfkeywords={蔡和章}
}

\begin{document}
\pagenumbering{gobble} % suppress displaying page number

\name{\bf 蔡\ 和章}

\basicInfo{
    \email{mail@me.com} \textperiodcentered\ 
    \phone{(+86) 123-4567-8900} \textperiodcentered\
    \homepage[blog.me]{https://cnblogs.com}
}

\section{\faGraduationCap\ 教育背景}
\datedsubsection{\textbf{五角场职业学院~} 上海}{2017.09 -- 2020.06}
\textit{硕士},信息与通信工程,GPA:3.53~/~4 \\[1pt]
研究方向:机器学习、信息论

\datedsubsection{\textbf{广埠屯家里蹲大学~} 湖北,武汉}{2013.09 -- 2017.06}
\textit{理学学士},信息与计算科学,GPA:3.721~/~5,专业排名:1~/~43

% programming sicp SICP The learning progress of sicp is partially 
% distributed on github. say some code \verb|int c++ iam Monaco| 
% and \textbf{some bold text} \textbf{\textit{some bold italic}} 
% \texttt{typeset} \textsf{Martel Sans} see \textsf{some 
% \textsc{Procedure} sans Procedure and \textit{sans font} and 
% \textbf{bolded sans}} \underline{this is underlined}

\section{\faUsers\ 工作经历}
\datedsubsection{\textbf{互联网大厂} ~杭州}{2020.08 -- 至今}
\role{XX工作室,XX岗位}{在职期间干了XX大事,做了XX贡献。}
\vspace{-4pt}
\begin{itemize}
  \item XX系统
  \begin{itemize}
    \item 解决了XX问题
    \item 提升了XX性能
    \item 实现了XX功能
  \end{itemize}
  \item 另一个XX系统
  \begin{itemize}
    \item 开发了XX框架
    \item 解决了XX痛点
  \end{itemize}
  \item 其他XX模块
\end{itemize}

\section{\faBriefcase 实习经历}
\datedsubsection{\textbf{游戏大厂} ~上海}{2019.07 -- 2019.09}
\role{服务器引擎部,游戏开发实习生}{学习了XX技能。}

\datedsubsection{\textbf{金融大厂} ~上海}{2019年夏}
\role{软件开发实习生}{实习期间解决了XX问题。}

\section{\faCogs\ 工作技能}
\begin{itemize}[parsep=0.5ex]
  \item 熟悉常用的算法和数据结构,如链表、栈和队列、树、堆等
  \item 熟悉C++基本特性,熟悉STL,如顺序容器、泛型算法
  \item 熟悉Python脚本语言,有相关开发经验
  \item 熟悉常用的机器学习算法,如支持向量机、Logistic回归、人工神经网络、决策树、K-近邻法等
  \item 熟悉常用的Linux命令,PC使用Linux系统已有3年多,有一定的Linux基础
  \item 通过国家英语六级,有较好的英文阅读能力
\end{itemize}

% \section{\faCode\ 求职意向}
% \begin{itemize}
%   \item C++开发
%   \item 数据挖掘、机器学习
%   \item Linux运维
% \end{itemize}

% \section{\faHeartO\ 获奖及荣誉}
% \begin{itemize}
% \end{itemize}

\section{~\faInfo~\ 其他信息}
\begin{itemize}[parsep=0.5ex]
  \item \datedline{基础信息论本科生助教}{2018年秋}
  \item \datedline{国际会议志愿者}{2018.07}
  \item \datedline{概率论与数理统计本科生助教}{2017年秋}
  \item \datedline{校级三好学生}{2015.11}
  \item \datedline{华中地区大学生数学建模邀请赛三等奖}{2015.05}
  \item \datedline{国家励志奖学金}{2014, 2015}
\end{itemize}

%% Reference
%\newpage
%\bibliographystyle{IEEETran}
%\bibliography{mycite}
\end{document}
