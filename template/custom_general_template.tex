\documentclass[10pt,a4paper,twocolumn]{article}
%===================================================
%= normalsize = 10pt,11pt,12pt 三个选项
%= onecolumn,twocolumn 单双栏设置
%===================================================

%====== 导言区 =====%

%=========== 自定义页边距 ===============%
\usepackage{geometry}
\geometry{left = 1cm, right = 1cm, top = 1cm, bottom = 0.7cm}

%========== math symbol ==============%
\usepackage{amsmath,amsfonts,amssymb}
% \usepackage{mathrsfs,calrsfs}   %花体字母支持

%============ other support ==============%
\usepackage{url}    %超链接支持
\usepackage{ctex}   %中文支持,推荐使用 xelatex 编译
\usepackage{enumerate}    %有序列表支持
\usepackage{graphicx}   %插图支持
\usepackage{tikz}   %tikz画图支持
\usepackage{xcolor}   %色彩支持

%================= 自定义区 ====================%
\linespread{1}    %行间距
% \parskip=10pt   %段间距

%================= 导入自定义命令 ===============%
\input defs.tex

\begin{document}

\section{Graph Algo}

\begin{itemize}
  \item 邻接矩阵:空间复杂度 \(O(n^2)\),判断(u,v)是否存在一条边需要常数时间,遍历所有边耗时 \(\Theta(n^2)\).
  \item 邻接表:空间复杂度 \(O(m+n)\),判断(u,v)是否存在一条边需要 \(O(deg(u))\)时间,遍历所有边耗时 \(\Theta(m+n)\).
  \item The distance between two nodes s and t is the length of the shortest path between them.
  \item{定理:}For each i, \(L_i\) consists of all nodes at distance exactly i from s. There is a path from s to t \(\iff\) t appears in some layer.
  \item{Porperty: }Let (x, y) be an edge of G. Then the level of x and y differ by at most 1.
  \item{BFS \& DFS: }
    % \begin{figure}[h]
    %   \includegraphics[scale=0.2]{cs2.png}
    %   \includegraphics[scale=0.2]{cs3.png}
    % \end{figure}
  \item{Theorem: }BFS/DFS runs in \(O(m + n)\) time if the graph is given by its adjacency list representation.
  \item Lemma. If a graph G is bipartite, it cannot contain an odd length cycle.
  \item{
    Lemma. Let G be a connected graph, and let \(L_0,\cdots,L_k\) be the layers produced by BFS starting at node s. Exactly one of the following holds.
    \begin{itemize}
      \item No edge of G joins two nodes of the same layer, and G is bipartite.
      \item An edge of G joins two nodes of the same layer, and G contains an odd-length cycle (and hence is not bipartite).
    \end{itemize}
  }
  \item Def. An DAG is a directed graph that contains no directed cycles.
  \item Def. A topological order of a directed graph G = (V, E) is an ordering of its nodes as \(v_1,v_2,\cdots,v_n\) so that for every edge \((v_i,v_j)\) we have \(i<j\).
  \item{Theorem. }G is a DAG. \(iff\) G has a topolical ordering.
  \item Lemma. If G is a DAG, then G has a node with no incoming edges.
  \item 找出G的入度为0的点,去掉它,在去掉后的图里再找入度为0的点。如此往复,去掉的节点顺序组成一个拓扑排序。复杂度为 \(O(m+n)\).
  \item 强连通性:正反BFS判断。
\end{itemize}

\section{Greedy Algo}
\begin{itemize}
  \item Interval Scheduling: Sort job by finsh time. Then greedy. Greedy is optimal.
  \begin{itemize}
    \item Assume greedy is not optimal, and let's see what happens.
    \item Let \(i_1,\cdots,i_k\) denote set of jobs selected by greedy.
    \item Let \(j_1,\cdots,j_m\) denote set of jobs in the optimal solution with \(i_1=j_1,\cdots,i_r=j_r\) for the largest possible value of r.
    \item 在这一步贪心解不必其他解差。故最优。
  \end{itemize}
  \item Interval Partitioning:
  \begin{figure}[h]
    \centering
    \includegraphics[scale=0.2]{IntervalPartitioning.png}
  \end{figure}
  \begin{itemize}
    \item Let d = number of classrooms that the greedy algorithm allocates.
    \item Classroom d is opened because we needed to schedule a job, say j, that is incompatible with all d-1 other classrooms.
    \item Since we sorted by start time, all these incompatibilities are caused by lectures that start no later than \(s_j\).
    \item Thus, we have d lectures overlapping at time \(s_j + \epsilon\).
    \item Key observation \(\implies\) all schedules use \(\ge\) d classrooms.
  \end{itemize}
  \begin{figure}[h]
    \centering
    \includegraphics[scale=0.2]{dijkstra.png}
  \end{figure}
  \item Proof Skills:
    \begin{itemize}
      \item Greedy algorithm stays ahead. Show that after each step of the greedy algorithm, its solution is at least as good as any other algorithm's.
      \item Structural. Discover a simple "structural" bound asserting that every possible solution must have a certain value. Then show that your algorithm always achieves this bound.
      \item Exchange argument. Gradually transform any solution to the one found by the greedy algorithm without hurting its quality.
    \end{itemize}
\end{itemize}

\section{Divide and Conquer}
\begin{figure}[h]
    \centering
    \includegraphics[scale=0.2]{karatsuba.png}
    \includegraphics[scale=0.2]{matrixMultiply.png}
\end{figure}
\begin{itemize}
  \item Merge sort: Divide a sequence into two of same size
  \item Counting inversionn: Extension of mergesort
  \item Closest Pair of Points: Vertically divide the space
  \item Fast Fourier Transform: Divide a polynomial into two with even and odd powers
\end{itemize}

\section{Dynamic Programming}
\begin{itemize}
  \item{\textbf{Divide and Conquer}}: Break up a problem into a few sub-problems, solve each sub-problem independently and recursively, and combine solution to sub-problems to form solution to original problem.
  \item{\textbf{Dynamic Programming}}: Break up a problem into a series of overlapping sub-problems, and build up solutions to larger and larger sub-problems.
  \item Weighted Interval Scheduling: Memoized version of algorithm takes \(O(n\log{n})\) time. 排序 \(O(n)\),计算 \(p(j)\) 耗时 \(O(n)\) after sorting by start time,函数调用 \(O(n)\).
  \begin{figure}[h]
    \centering
    \includegraphics[scale=0.2]{weightedIS.png}
    \caption{Weighted Interval Scheduling}
    \includegraphics[scale=0.2]{knapspack.png}
    \caption{Knapspack}
    \includegraphics[scale=0.2]{sequenceAlign.png}
    \caption{Sequence Alignment}
    \includegraphics[scale=0.2]{bellman-ford.png}
    \caption{Bellman-Ford Algo}
  \end{figure}
  \item Kanpspack Problem: \(O(nW)\) 伪多项式,因为问题规模随着  \(\log_2{W}\) 线性增长。
  \item 负环
  \begin{itemize}
    \item If OPT(n,v) = OPT(n-1,v) for all v, then there is no negative cycle with a path to t.
    \item If OPT(n,v) < OPT(n-1,v) for some node v, then (any) shortest path from v to t contains a cycle W. Moreover W has negative cost.
    \begin{figure}[h]
    \centering
    \includegraphics[scale=0.2]{negativeCycle.png}
  \end{figure}
  \end{itemize}
\end{itemize}

\section{Max Flow}
\begin{itemize}
  \item An s-t cut is a partition (A, B) of V with \(s \in A\) and \(t \in B\).
  \item Flow value lemma. Let f be any flow, and let (A, B) be any s-t cut. Then, the net flow sent across the cut is equal to the amount leaving s.
  \item Weak duality. Let f be any flow, and let (A, B) be any s-t cut. Then the value of the flow is at most the capacity of the cut.
  \item Augmenting path theorem. Flow f is a max flow iff there are no augmenting paths.
  \item Max-flow min-cut theorem. The value of the max flow is equal to the value of the min cut.
  \begin{figure}[h]
    \centering
    \includegraphics[scale=0.2]{property2tui3.png}
    \includegraphics[scale=0.2]{min-cut.png}
  \end{figure}
\end{itemize}


\end{document}
