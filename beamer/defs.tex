%%%%%%%%%%%%%%%%%%%%%%%%%%%%%%%%%%%%%%%%%%%%%%%
% 1. This is a definition file
% 2. use it by insert `\input defs.tex` in your document (导言区)
% 3. covers some frequently used notation in probability & convex analysis
%
% comment by yychi @ 2018.06.03
%%%%%%%%%%%%%%%%%%%%%%%%%%%%%%%%%%%%%%%%%%%%%%%%%
\newcommand{\ones}{\mathbf 1}
\newcommand{\zo}{\mathbf 0}
\newcommand{\reals}{{\mbox{\bf R}}}
\newcommand{\integers}{{\mbox{\bf Z}}}
\newcommand{\symm}{{\mbox{\bf S}}}  % symmetric matrices

\newcommand{\nullspace}{{\mathcal N}}
\newcommand{\range}{{\mathcal R}}
\newcommand{\Rank}{\mathop{\bf Rank}}
\newcommand{\Tr}{\mathop{\bf Tr}}
\newcommand{\diag}{\mathop{\bf diag}}
\newcommand{\card}{\mathop{\bf card}}
\newcommand{\SNR}{\textrm{SNR}}
\newcommand{\rank}{\mathop{\bf rank}}
\newcommand{\conv}{\mathop{\bf conv}}
\newcommand{\prox}{\mathbf{prox}}

\newcommand{\E}{\mathbb{E}{}}
\newcommand{\Prob}{\mathop{\bf Prob}}
\newcommand{\Co}{{\mathop {\bf Co}}} % convex hull
\newcommand{\dist}{\mathop{\bf dist{}}}
\newcommand{\argmin}{\mathop{\rm argmin}}
\newcommand{\argmax}{\mathop{\rm argmax}}
\newcommand{\epi}{\mathop{\bf epi}} % epigraph
\newcommand{\Vol}{\mathop{\bf vol}}
\newcommand{\dom}{\mathop{\bf dom}} % domain
\newcommand{\inte}{\mathop{\bf int}}
\newcommand{\bound}{\mathop{\bf boundary}}
\newcommand{\sign}{\mathop{\bf sign}}
\newcommand{\mysign}[1]{\mathop{\rm{#1}}}
\newcommand{\question}[1]{{\color{red}{#1}}}
% \renewcommand{\det}[1]{|{#1}|}

\newcommand{\cf}{{\it cf.}}
\newcommand{\eg}{{\it e.g.}}
\newcommand{\ie}{{\it i.e.}}
\newcommand{\etc}{{\it etc.}}

\definecolor{color1}{HTML}{D0B22B}
\definecolor{dred}{RGB}{128,0,0}
\definecolor{colorhkust}{RGB}{20,43,140}
\definecolor{colorshanghaitech}{RGB}{162,0,5}
\definecolor{colortsinghua}{RGB}{116,52,129}
\definecolor{colorpink}{RGB}{255,204,204}
\definecolor{colororange}{RGB}{255,153,102}
\definecolor{header1}{cmyk}{.9,.5,0,.35}
\newcommand{\ep}[1]{{\bf\color{colorhkust}{#1}}}
\newcommand{\epeq}[1]{{\mathbf\color{colorhkust}{#1}}}
\newcommand{\epb}[1]{{\bf\color{colorshanghaitech}{#1}}}

\newcommand{\defpoints}[1]{{\color{magenta}{(#1 points)}}}
\newcommand{\sol}{{\textcolor{colortsinghua}{Solution: }}}
\newcommand{\bd}[1]{\boldsymbol{#1}}
